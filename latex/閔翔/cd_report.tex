\documentclass[14pt,a4paper]{report}  %紙張設定
\usepackage{xeCJK}%中文字體模組
\setCJKmainfont{標楷體} %設定中文字體
%\setCJKmainfont{MoeStandardKai.ttf}
\newfontfamily\sectionef{Times New Roman}%設定英文字體
%\newfontfamily\sectionef{Nimbus Roman}
\usepackage{enumerate}
\usepackage{amsmath,amssymb}%數學公式、符號
\usepackage{amsfonts} %數學簍空的英文字
\usepackage{graphicx, subfigure}%圖形
\usepackage{fontawesome5} %引用icon
\usepackage{type1cm} %調整字體絕對大小
\usepackage{textpos} %設定文字絕對位置
\usepackage[top=2.5truecm,bottom=2.5truecm,
left=3truecm,right=2.5truecm]{geometry}
\usepackage{titlesec} %目錄標題設定模組
\usepackage{titletoc} %目錄內容設定模組
\usepackage{textcomp} %表格設定模組
\usepackage{multirow} %合併行
%\usepackage{multicol} %合併欄
\usepackage{CJK} %中文模組
\usepackage{CJKnumb} %中文數字模組
\usepackage{wallpaper} %浮水印
\usepackage{listings} %引用程式碼
\usepackage{hyperref} %引用url連結
\usepackage{setspace}
\usepackage{lscape}%設定橫式
\lstset{language=Python, %設定語言
		basicstyle=\fontsize{10pt}{2pt}\selectfont, %設定程式內文字體大小
		frame=lines,	%設定程式框架為線
}
%\usepackage{subcaption}%副圖標
\graphicspath{{./../images/}} %圖片預設讀取路徑
\usepackage{indentfirst} %設定開頭縮排模組
\renewcommand{\figurename}{\Large 圖.} %更改圖片標題名稱
\renewcommand{\tablename}{\Large 表.}
\renewcommand{\lstlistingname}{\Large 程式.} %設定程式標示名稱
\hoffset=-5mm %調整左右邊界
\voffset=-8mm %調整上下邊界
\setlength{\parindent}{3em}%設定首行行距縮排
\usepackage{appendix} %附錄
\usepackage{diagbox}%引用表格
\usepackage{multirow}%表格置中
%\usepackage{number line}
%=------------------更改標題內容----------------------=%
\titleformat{\chapter}[hang]{\center\sectionef\fontsize{20pt}{1pt}\bfseries}{\LARGE 第\CJKnumber{\thechapter}章}{1em}{}[]
\titleformat{\section}[hang]{\sectionef\fontsize{18pt}{2.5pt}\bfseries}{{\thesection}}{0.5em}{}[]
\titleformat{\subsection}[hang]{\sectionef\fontsize{18pt}{2.5pt}\bfseries}{{\thesubsection}}{1em}{}[]
%=------------------更改目錄內容-----------------------=%
\titlecontents{chapter}[11mm]{}{\sectionef\fontsize{18pt}{2.5pt}\bfseries\makebox[3.5em][l]
{第\CJKnumber{\thecontentslabel}章}}{}{\titlerule*[0.7pc]{.}\contentspage}
\titlecontents{section}[18mm]{}{\sectionef\LARGE\makebox[1.5em][l]
{\thecontentslabel}}{}{\titlerule*[0.7pc]{.}\contentspage}
\titlecontents{subsection}[4em]{}{\sectionef\Large\makebox[2.5em][l]{{\thecontentslabel}}}{}{\titlerule*[0.7pc]{.}\contentspage}
%=----------------------章節間距----------------------=%
\titlespacing*{\chapter} {0pt}{0pt}{18pt}
\titlespacing*{\section} {0pt}{12pt}{6pt}
\titlespacing*{\subsection} {0pt}{6pt}{6pt}
%=----------------------標題-------------------------=%             
\begin{document} %文件
\sectionef %設定英文字體啟用
\vspace{12em}
\begin{titlepage}%開頭
\begin{center}   %標題  
\makebox[1.5\width][s] %[s] 代表 Stretch the interword space in text across the entire width
{\fontsize{24pt}{2.5pt}國立虎尾科技大學}\\[18pt]
\makebox[1.5\width][s]
{\fontsize{24pt}{2.5pt}機械設計工程系}\\[18pt]
\sectionef\fontsize{24pt}{1em}\selectfont\textbf
{
\vspace{0.5em}
cd2023 2a-pj1ag1分組報告}\\[18pt]
%設定文字盒子 [方框寬度的1.5倍寬][對其方式為文字平均分分布於方框中]\\距離下方18pt
\vspace{1em} %下移
\fontsize{30pt}{1pt}\selectfont\textbf{網際足球泡泡機器人場景設計}\\
\vspace{1em}
\sectionef\fontsize{30pt}{1em}\selectfont\textbf
{
\vspace{0.5em}
Web-based bubbleRob Football Scene Design}
 \vspace{2em}
%=---------------------參與人員-----------------------=%             
\end{center}
\begin{flushleft}
\begin{LARGE}

\hspace{32mm}\makebox[5cm][s]
{指導教授:\quad 嚴\quad 家\quad 銘\quad 老\quad 師}\\[6pt]
\hspace{32mm}\makebox[5cm][s]
{班\qquad 級:\quad 四\quad 設\quad 二\quad 甲}\\[6pt]
\hspace{32mm}\makebox[5cm][s]
{學\qquad 生:\quad 紀\quad 閔\quad 翔\quad(41023147)}
\\[6pt]
\hspace{32mm}\makebox[5cm][s]
{\hspace{36.5mm}施\quad 建\quad 菖\quad(41023143)}\\[6pt]

%設定文字盒子[寬度為5cm][對其方式為文字平均分分布於方框中]空白距離{36.5mm}\空白1em
\end{LARGE}
\end{flushleft}
\vspace{6em}
\fontsize{18pt}{2pt}\selectfont\centerline{\makebox[\width][s]
{中華民國\hspace{3em} 
112 \quad 年\quad 3\quad 月}}
\end{titlepage}
\newpage


%=------------------------摘要-----------------------=%
\renewcommand{\baselinestretch}{1.5} %設定行距
\pagenumbering{roman} %設定頁數為羅馬數字
\clearpage  %設定頁數開始編譯
\sectionef
\addcontentsline{toc}{chapter}{摘~~~要} %將摘要加入目錄
\begin{center}
\LARGE\textbf{摘~~要}\\
\end{center}
\begin{flushleft}
\fontsize{14pt}{20pt}\sectionef\hspace{12pt}\quad 本課程將在設計簡單的移動機器人bubbleRob的同時,介紹相當多的CoppeliaSim功能。本教程相關的 CoppeliaSim 場景文件位於scenes/tutorials/BubbleRob。scenes/tutorials/BubbleRob。\\[12pt]

\fontsize{14pt}{20pt}\sectionef\hspace{12pt}\quad 此專題是運用足球機器人,將其導入 CoppeliaSim 模擬環境並給予對應設置,將其機電系統簡化並運用 AI 進行訓練,找到適合此系統的演算法後,再到 CoppeliaSim 模擬環境中進行測試演算法在實際運用上
的可行性。並嘗試透過架設伺服器將 CoppeliaSim 影像串流到網頁供使用者觀看或操控。
\\[12pt]

\end{flushleft}
\begin{center}
\fontsize{14pt}{20pt}\selectfont 關鍵字: 類神經網路、強化學習、\sectionef 、Chat GPT、CoppeliaSim、OpenAI Gym
\end{center}
\newpage
%=--------------------Abstract----------------------=%
\renewcommand{\baselinestretch}{1.5} %設定行距
\addcontentsline{toc}{chapter}{Abstract} %將摘要加入目錄
\begin{center}
\LARGE\textbf\sectionef{Abstract}\\
\begin{flushleft}
\fontsize{14pt}{16pt}\sectionef\hspace{12pt}\quad Due to the four major development trends of multidimensional arrays  computing, automatic differentiation, open source development environment, and multi-core GPUs computing hardware. The rapid development of the AI field has been promoted. In view of this development, the physical mechatronic systems can gain machine learning efficiency through their simulated virtual system training process. And afterwards to apply the trained model into real mechatronic systems.\\[12pt]

\fontsize{14pt}{16pt}\sectionef\hspace{12pt}\quad This project is to use the physical air hockey to play machine, introduce it into the CoppeliaSim simulation environment and give the corresponding settings, simplify its electromechanical system and use Open AI Gym for training, find an algorithm suitable for this system, and then perform it in the CoppeliaSim simulation environment Feasibility of testing algorithm in practical application. And try to stream CoppeliaSim images to web pages for users to watch or manipulate by setting up a server.\\
\end{flushleft}
\begin{center}
\fontsize{14pt}{16pt}\selectfont\sectionef Keyword:  nerual network、reinforcement learning、 CoppeliaSim、OpenAI Gym
\end{center}
\newpage
%=------------------------誌謝----------------------=%
\addcontentsline{toc}{chapter}{誌~~~謝}
\centerline\LARGE\textbf{誌~~謝}\\
\begin{flushleft}
\fontsize{14pt}{2.5pt}\hspace{12pt}\quad 在此鄭重感謝製作以及協助本分組報告完成的所有人員,首先向大三學長致謝,他們不辭辛勞解決我們的提問,甚至從來沒有不耐煩,總是貼心為我們找出最佳解答。再來是我們的分組組長,他給了我們全方位的支援,提供我們解決問題的方向和建議,給予開始接觸網際對戰遊戲的我們有個學習的方向,開會時也時不時向我們提出建議以及未來走向,同時也給了我們能自由摸索的空間及時間,最後是由本分組組員同心協力才得以完成本報告,特此感謝。
\end{flushleft}
\newpage
%=------------------------目錄----------------------=%
\renewcommand{\contentsname}{\centerline{\fontsize{18pt}{\baselineskip}\selectfont\textbf{目\quad 錄}}}
\tableofcontents  %目錄產生


\end{center}
%=-------------------------內容----------------------=%
\chapter{前言}
\renewcommand{\baselinestretch}{10.0} %設定行距
\pagenumbering{arabic} %設定頁號阿拉伯數字
\setcounter{page}{1}  %設定頁數
\fontsize{14pt}{2.5pt}\sectionef
\section{設計架構}
此次 pj3 專題目標有建立場景中的計時器、球員外型及移動優化、添加球員擊球和翻車再起技能、進球後收集並隨機投下新的一顆球、建立以機械式轉盤傳動計分系統。由於目標繁多,需要組員間分工負責,在每個禮拜的協同中逐步完成 pj3 專題。\\

\begin{figure}[hbt!]
\begin{center}
\includegraphics[width=10cm]{設計目標}
\caption{\Large 設計架構圖}\label{fig.設計目標}
\end{center}
\end{figure}

\section{規則說明}
類似於足球遊戲,一開始時球會置於場中央,遊戲開始後兩方即可以鍵盤操控機器人推球至已方的球門得分。\\
遊戲規則如下:
\begin{enumerate}
\item 球觸碰到情們感測器即算得分。
\item 最快獲得5分的隊伍即獲勝。
\item 任一方進球得分後,遊戲會重置,雙方回到場中央重新開球。
\end{enumerate}

\renewcommand{\baselinestretch}{0.5} %設定行距
\newpage
\section{製作過程}
1.首先繪製球框。\
\includegraphics[angle=0,width=10cm]{2023-03-27}\\
\\2.這是lua腳本控制是bubbleRob前後左右sim.getObjectHandle這個函式在coppeliasim4.3.0版本被淘汰了但還是可以使用,但用sim.getObject會比較好在後面感測器腳本已做改善,這個程式是用上下左右控制bubbleRob空白鍵暫停開始。\
\begin{center}
\includegraphics[angle=0,width=9cm]{螢幕擷取畫面 2023-04-14 221556}\\
\end{center}
 \newpage
3.加入球框感測器和記分板。\
\begin{center}
\includegraphics[angle=0,width=15cm]{football}
\end{center}

詳情可見\\
\href{https://mdecd2023.github.io/football-apj1/content/ag2.html}{https://mdecd2023.github.io/football-apj1/content/ag2.html}\\
\newpage
\begin{figure}[hbt!]
\chapter{環境設定}
\section{碰撞檢測}
\end{figure}
1.可探測(Detectable),可讓感測器感測到物體。在本遊戲中,球應該將Detectable打開,這樣才可以感測到進球,而bubbleRob機器人則要把Detectable關掉,不然機器人碰到感測器也會得分。\
\begin{center}
\includegraphics[angle=0,width=12cm]{48.1}\\
\includegraphics[angle=0,width=12cm]{48.2}
\end{center}
\newpage
2.在球框前加入射線感測器(Ray type),這樣球進框就一定會碰到感測器,射線感測器不能貼地不然感測不到。
\begin{center}
\includegraphics[angle=0,width=10cm]{48.5}
\end{center}

3.球框及圍牆的Body is respondable要打開不然球會穿過去,而Body is dynamic則要關上不然球框會亂動。
\begin{center}
\includegraphics[angle=0,width=10cm]{48.3}
\end{center}

4.開啟Connectivity->Visualization,從http://127.0.0.1:23020/ 中可看見。
\begin{center}
\includegraphics[angle=0,width=10cm]{4810}
\end{center}
%=---------------------參考文獻----------------------=%
\addcontentsline{toc}{chapter}{參考文獻} %新增目錄名稱
\newpage
\renewcommand\bibname{參~考~文~獻}
\begin{thebibliography}{99}  % 參考文獻印出之編號最寬為兩個字母寬
\bibitem 1\href{https://www.coppeliarobotics.com/helpFiles/index.html}{https://www.coppeliarobotics.com/helpFiles/index.html}
\end{thebibliography}
%=---------------附錄-----------------=%
\addcontentsline{toc}{chapter}{附錄} %新增目錄名稱
\input{10_appendix.tex}



%\newpage
%\begin{landscape}  %橫式環境
%\begin{center}
%\fontsize{0.001pt}{1pt}\selectfont .
%\vspace{70mm}
%\rotatebox[origin=cc]{90}{\LARGE 【14】}\rotatebox[origin=cc]%{180}{\LARGE 1-2-APP-8765} %旋轉
%\end{center}
%\end{landscape}
\end{document}
