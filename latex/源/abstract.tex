\renewcommand{\baselinestretch}{1.5} %設定行距
\pagenumbering{roman} %設定頁數為羅馬數字
\clearpage  %設定頁數開始編譯
\sectionef
\addcontentsline{toc}{chapter}{摘~~~要} %將摘要加入目錄
\begin{center}
\LARGE\textbf{摘~~要}\\
\end{center}
\begin{flushleft}
\fontsize{14pt}{20pt}\sectionef\hspace{12pt}\quad 本課程將在設計簡單的移動機器人BubbleRob的同時,介紹相當多的 CoppeliaSim 功能。本教程相關的 CoppeliaSim 場景文件位於scenes/tutorials/BubbleRob。\\[12pt]

\fontsize{14pt}{20pt}\sectionef\hspace{12pt}\quad 此專題是運用足球機器人,將其導入CoppeliaSim模擬環境並給予對應設置,將其機電系統簡化並運用 AI 進行訓練,找到適合此系統的演算法後,再到CoppeliaSim模擬環境中進行測試演算法在實際運用上的可行性。並嘗試透過架設伺服器將CoppeliaSim影像串流到網頁供使用者觀看或操控。\\[12pt]

\end{flushleft}
\begin{center}
\fontsize{14pt}{20pt}\selectfont 關鍵字: 類神經網路、強化學習、caht gpt、\sectionef CoppeliaSim、OpenAI Gym
\end{center}