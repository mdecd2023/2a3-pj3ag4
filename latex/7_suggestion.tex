\chapter{總結}
本學期透過協同合作之方式,從pj1的雙人協同,完成了tutorial1之BubbleRob設計,且在BubbleRob中加上了感測器,載入了lua程式碼,使其擁有避開障礙物之功能。隨後對其加上zmq程式碼使BubbleRob能在CoppeliaSim中由操作者自由移動。接著進入了四人的pj2協同,設計出記分板並加入進球場中,即可進行雙人之足球對戰。再來進行了八人為一組的協同,每位組員分工進行對球場與球員BubbleRob進行改善修正bug,使其對戰更為完善。\\

協同合作可以讓我們對一個項目的開發效率提高,組員可以針對自己的專長與能力來進行任務分配。且協同合作具有共享資源特性,組員之間可以互相學習和激發出創新的想法。隨著科技在不斷的進步,我們為了跟上世界的發展,利用協同合作產品開發的方式是必不可少的。\\